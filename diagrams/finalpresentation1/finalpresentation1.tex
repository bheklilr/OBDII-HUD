\documentclass{standalone}
  \usepackage{tikz}
  \usepackage[none]{hyphenat}
  \usetikzlibrary{positioning,shapes,fit,calc,shadows,arrows}

\begin{document}

\pgfdeclarelayer{background}
\pgfdeclarelayer{foreground}
\pgfsetlayers{background,main,foreground}

\tikzstyle{block}=[rectangle, draw=black, rounded corners, text=black, rectangle split, rectangle split parts=3, text centered, anchor=north, rectangle split draw splits=false]

\newcommand{\fntsize}[1]{\fontsize{#1}{15}\selectfont}
\newcommand{\blocktitle}[1]{\fntsize{16}\textbf{#1}}
\newcommand{\blocktext}[1]{\nodepart{second}\fntsize{14}#1}

\begin{tikzpicture}
  \node (toplevel) [block, text width=9in, fill=blue!15] {
    \blocktitle{OBDII-HUD}
    \blocktext{Top-level package containing the libraries and scripts needed to display OBDII data}
    \nodepart{third}
      \begin{tikzpicture}[node distance=0.125in]
        \node (obd2) [block, text width=8in, fill=blue!35] {
          \blocktitle{OBDII}
          \blocktext{Implementation of communications protocols and data parsing for the OBDII sensor}
          \nodepart{third}
            \begin{tikzpicture}[node distance=0.25in]
              \node (data) [block, text width=3.5in, fill=blue!15, rectangle split parts=2, anchor=east, xshift=-0.125in] {
                \blocktitle{Data}
                \blocktext{Parses OBDII data into useful and meaningful data structures}
              };
              \node (comm) [block, text width=3.5in, fill=blue!15, anchor=west, xshift=0.125in] {
                \blocktitle{Communcations}
                \blocktext{Handles the IO with the OBDII sensor}
                \nodepart{third}
                  \begin{tikzpicture}[node distance=0.125in]
                    \node (obd2bt) [block, text width=3in, fill=blue!35, rectangle split parts=2] {
                      \blocktitle{OBDII Bluetooth}
                      \blocktext{Connects to the OBDII sensor via Bluetooth}
                    };
                    \node (obd2usb) [block, text width=3in, fill=blue!35, rectangle split parts=2, below=of obd2bt] {
                      \blocktitle{OBDII USB}
                      \blocktext{Connects to the OBDII sensor via USB}
                    };
                  \end{tikzpicture}
              };
            \end{tikzpicture}
        };

        \node (display) [block, text width=8in, fill=blue!35, below=of obd2] {
          \blocktitle{Display}
          \blocktext{Transforms the parsed data into graphs and displays for the user}
          \nodepart{third}
            \begin{tikzpicture}[node distance=0.25in]
              \node (gui) [block, text width=3.5in, fill=blue!15, rectangle split parts=2, anchor=east, xshift=-0.125in] {
                \blocktitle{GUI}
                \blocktext{Implementation the graphic user interface framework that can be dynamically generated from the }settings
              };
              \node (core) [block, text width=3.5in, fill=blue!15, anchor=west, xshift=0.125in] {
                \blocktitle{Core}
                \blocktext{Contains higher level drawing functions to create more complicated interfaces}
                \nodepart{third}
                  \begin{tikzpicture}[node distance=0.25in]
                    \node (drawing) [block, text width=3in, fill=blue!35, rectangle split parts=2] {
                      \blocktitle{Drawing}
                      \blocktext{Implements resusable and simple drawing methods deriving from the graphics engine}
                    };
                  \end{tikzpicture}
              };
            \end{tikzpicture}
        };

        \node (settings) [block, text width=8in, fill=blue!35, below=of display] {
          \blocktitle{Settings}
          \blocktext{Implementation of dynamic settings that can be used to construct the interface and select desired data}
          \nodepart{third}
            \begin{tikzpicture}
              \node (parser) [block, text width=6in, fill=blue!15, rectangle split parts=2] {
                \blocktitle{Parser}
                \blocktext{The JSON backend for the settings manager}
              };
            \end{tikzpicture}
        };
      \end{tikzpicture}
  };
\end{tikzpicture}

\end{document}
