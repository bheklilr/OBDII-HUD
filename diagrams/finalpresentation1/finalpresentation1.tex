\documentclass{minimal}
\usepackage[a4paper,margin=1cm, landscape]{geometry}
\usepackage{tikz}
\usepackage[none]{hyphenat}
\usetikzlibrary{positioning,shapes,fit,calc,shadows,arrows}

\begin{document}

\pgfdeclarelayer{background}
\pgfdeclarelayer{foreground}
\pgfsetlayers{background,main,foreground}

\tikzstyle{block}=[rectangle, draw=black, rounded corners, text=black, rectangle split, rectangle split parts=3, text centered, anchor=north, rectangle split draw splits=false]

\begin{center}

    \begin{tikzpicture}[node distance=0.25in]
        \node (toplevel) [block, text width=8in] {
            \textbf{OBDII-HUD}
            \nodepart{second} Software to connect to an OBDII sensor and display real-time information on the windshield of a vehicle.
            \nodepart{third}
                \begin{tikzpicture}[node distance=0.25in]
                    \node (obd2) [block, text width=7in, fill=blue!35] {
                        \textbf{OBDII}
                        \nodepart{second} Controls the connection to the OBDII sensor and parses output
                        \nodepart{third}
                            \begin{tikzpicture}[node distance=0.25in]
                                \node (data) [block, text width=2.5in, fill=blue!15, rectangle split parts=2, anchor=east, xshift=-0.125in] {
                                    \textbf{Data}
                                    \nodepart{second} Parses data into useable structures
                                };
                                \node (comm) [block, text width=3in, fill=blue!15, anchor=west, xshift=0.125in] {
                                    \textbf{Communcations}
                                    \nodepart{second} Handles the IO with the OBDII sensor
                                    \nodepart{third}
                                        \begin{tikzpicture}[node distance=0.125in]
                                            \node (obd2bt) [block, text width=2.5 in, fill=blue!35, rectangle split parts=2] {
                                                \textbf{OBDII Bluetooth}
                                                \nodepart{second}Connects to the OBDII sensor via Bluetooth
                                            };
                                            \node (obd2usb) [block, text width=2.5in, fill=blue!35, rectangle split parts=2, below=of obd2bt] {
                                                \textbf{OBDII USB}
                                                \nodepart{second} Connects to the OBDII sensor via USB
                                            };
                                        \end{tikzpicture}
                                };
                            \end{tikzpicture}
                    };

                    \node (display) [block, text width=7in, fill=blue!35, below=of obd2] {
                        \textbf{Display}
                        \nodepart{second}Transforms the raw data into meaningful graphs.
                        \nodepart{third}
                            \begin{tikzpicture}[node distance=0.25in]
                                \node (gui) [block, text width=2.5in, fill=blue!15, rectangle split parts=2, anchor=east, xshift=-0.125in] {
                                    \textbf{GUI}
                                    \nodepart{second} Implementation the graphic user interface framework that can be dynamically generated from the settings
                                };
                                \node (core) [block, text width=3in, fill=blue!15, anchor=west, xshift=0.125in] {
                                    \textbf{Core}
                                    \nodepart{second} Contains higher level drawing functions to create complicated interfaces
                                    \nodepart{third}
                                        \begin{tikzpicture}[node distance=0.25in]
                                            \node (drawing) [block, text width=2.5in, fill=blue!35, rectangle split parts=2] {
                                                \textbf{Drawing}
                                                \nodepart{second} Implements resusable and simple drawing methods deriving from the graphics engine
                                            };
                                        \end{tikzpicture}
                                };
                            \end{tikzpicture}
                    };

                    \node (settings) [block, text width=7in, fill=blue!35, below=of display] {
                        \textbf{Settings}
                        \nodepart{second}Implementation of dynamic settings that can be used to construct the interface
                        \nodepart{third}
                            \begin{tikzpicture}
                                \node (parser) [block, text width=6in, fill=blue!15, rectangle split parts=2] {
                                    \textbf{Parser}
                                    \nodepart{second}The JSON backend for the settings manager
                                };
                            \end{tikzpicture}
                    };
                \end{tikzpicture}
        };
        \begin{pgfonlayer}{background}
            \path (toplevel.west |- toplevel.north) + (-0.5, 0.5) node (a) {};
            \path (toplevel.east |- toplevel.south) + (0.5, -0.5) node (b) {};
            \path [fill=blue!15, rounded corners, draw=black!50] (a) rectangle (b);
        \end{pgfonlayer}
    \end{tikzpicture}

\end{center}

\end{document}
